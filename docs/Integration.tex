% Options for packages loaded elsewhere
\PassOptionsToPackage{unicode}{hyperref}
\PassOptionsToPackage{hyphens}{url}
%
\documentclass[
]{article}
\usepackage{lmodern}
\usepackage{amssymb,amsmath}
\usepackage{ifxetex,ifluatex}
\ifnum 0\ifxetex 1\fi\ifluatex 1\fi=0 % if pdftex
  \usepackage[T1]{fontenc}
  \usepackage[utf8]{inputenc}
  \usepackage{textcomp} % provide euro and other symbols
\else % if luatex or xetex
  \usepackage{unicode-math}
  \defaultfontfeatures{Scale=MatchLowercase}
  \defaultfontfeatures[\rmfamily]{Ligatures=TeX,Scale=1}
\fi
% Use upquote if available, for straight quotes in verbatim environments
\IfFileExists{upquote.sty}{\usepackage{upquote}}{}
\IfFileExists{microtype.sty}{% use microtype if available
  \usepackage[]{microtype}
  \UseMicrotypeSet[protrusion]{basicmath} % disable protrusion for tt fonts
}{}
\makeatletter
\@ifundefined{KOMAClassName}{% if non-KOMA class
  \IfFileExists{parskip.sty}{%
    \usepackage{parskip}
  }{% else
    \setlength{\parindent}{0pt}
    \setlength{\parskip}{6pt plus 2pt minus 1pt}}
}{% if KOMA class
  \KOMAoptions{parskip=half}}
\makeatother
\usepackage{xcolor}
\IfFileExists{xurl.sty}{\usepackage{xurl}}{} % add URL line breaks if available
\IfFileExists{bookmark.sty}{\usepackage{bookmark}}{\usepackage{hyperref}}
\hypersetup{
  pdftitle={Integration},
  pdfauthor={Euan Enticott},
  hidelinks,
  pdfcreator={LaTeX via pandoc}}
\urlstyle{same} % disable monospaced font for URLs
\usepackage[margin=1in]{geometry}
\usepackage{graphicx,grffile}
\makeatletter
\def\maxwidth{\ifdim\Gin@nat@width>\linewidth\linewidth\else\Gin@nat@width\fi}
\def\maxheight{\ifdim\Gin@nat@height>\textheight\textheight\else\Gin@nat@height\fi}
\makeatother
% Scale images if necessary, so that they will not overflow the page
% margins by default, and it is still possible to overwrite the defaults
% using explicit options in \includegraphics[width, height, ...]{}
\setkeys{Gin}{width=\maxwidth,height=\maxheight,keepaspectratio}
% Set default figure placement to htbp
\makeatletter
\def\fps@figure{htbp}
\makeatother
\setlength{\emergencystretch}{3em} % prevent overfull lines
\providecommand{\tightlist}{%
  \setlength{\itemsep}{0pt}\setlength{\parskip}{0pt}}
\setcounter{secnumdepth}{-\maxdimen} % remove section numbering

\title{Integration}
\author{Euan Enticott}
\date{16/12/2020}

\begin{document}
\maketitle

\hypertarget{integration}{%
\subsection{Integration}\label{integration}}

\hypertarget{quadrature}{%
\subsubsection{Quadrature}\label{quadrature}}

Quadratue rules are essentially integral approximations that use a
finite number of function evaluations. We will focus on approximating
the definite integral of \(f\) over \([a,b]\). For one dimensional
integrals you can use \texttt{integrate} in R. For multiple integrals
the \texttt{cubature} package can be used. These should provide more
robust implementations.

\hypertarget{polynomial-interpolation}{%
\subsubsection{Polynomial
interpolation}\label{polynomial-interpolation}}

The \textbf{Weierstrass Approximation Theorem} states that for
\(f \in C^{0}([a,b])\) there is a sequence of polynomials \((p_{n})\)
that converge uniformly to \(f\) on \([a,b]\). This suggests that for a
given tolerence there exists a polynomial that can be used to
approximate a given \(f\).

\hypertarget{lagrange-polynomials}{%
\subsubsection{Lagrange polynomials}\label{lagrange-polynomials}}

An approximation can be done useing lagrange polynomials. Consider
approximating \(f\) using \(k\) points
\(\{(x_{i}, f(x_{i}))\}_{i=1}^{k}\). The interpolating polynomial is
unique and has degree at most \(k-1\), it can be expressed as a Lagrange
polynomial: \[p_{k-1}(x) := \sum_{i=1}^{k}l_{i}(x)f(x_{i}),\] where the
Lagrange polynomials are:
\[l_{i}(x) = \prod_{j=1,j \neq i}\frac{x - x_{j}}{x_{i} - x_{j}} . \]

\hypertarget{polynomial-interpolation-error}{%
\subsubsection{Polynomial interpolation
error}\label{polynomial-interpolation-error}}

If \(f\) has \(k\) continuous derivatives and \(p_{k=1}\) is the
polynomial interpolating \(f\) at \(k\) points \(x_{1},...,x_{k}\) then
for any \(x \in [a,b]\) there exists \(\xi \in (a,b)\) such that:

\[f(x) - p_{k-1}(x) = \frac{1}{k!}f^{k}(\xi)\prod_{i=1}^{k}(x-x_{i})\]
This does not immediately provide a guarantee that a sequence of
interpolating polynomials will converge uniformly or pointwise to \(f\).
There is however a sequence of interpolation points that guarantee
uniform convergence. These are called Chebyshev points:
\[ \frac{cos(2i-1)}{2k}\pi, \; i \in \{1,...,k \},\] This bounds the
absolute value of the product term by \(2^{1-k}\). There still exist
functions that diverge on Chebyshev points but these are usually quite
complex. These only apply to differentiable functions.

\hypertarget{composite-polynomial-interpolation}{%
\subsubsection{Composite polynomial
interpolation}\label{composite-polynomial-interpolation}}

An alternative approach to fitting a high degree polynomial to many
points is to fit lower degree polynomials on subintervals of the domain.
This results in a piecewise polynomial approximation. For simplicity we
consider the region \([a,b]\) partitioned into equal length
subintervals. The scheme is `closed' if we inclue the endpoints and
`open' if we do not. Some examples are given below. For a large number
of subintervals the product term in the Interpolation Error can be made
very small.

\hypertarget{other-polynomial-interpolation-schemes}{%
\subsubsection{Other polynomial interpolation
schemes}\label{other-polynomial-interpolation-schemes}}

There are more polynomial interpolation schemes. You can use polynomials
that incorporate derivatives of \(f\) as well as \(f\), this is known as
Hermite interpolation. Matching derivatives at the boundaries of
subintervals in piecewise polynomial interpolation is called spline
interpolation. Cubic splines are a very popular way of approximating a
function.

\hypertarget{polynomial-integration}{%
\subsubsection{Polynomial integration}\label{polynomial-integration}}

We consider approximating the integral: \[I(f) := \int_{a}^{b}f(x)dx, \]
where \(f \in C^{0}([a,b])\). All approximations that we consider
involve computing integrals associated with polynomial approximations
such as we have considered above. The approximations are often referred
to as \emph{quadrature rules}.

\hypertarget{changing-limits-of-integration}{%
\paragraph{Changing limits of
integration}\label{changing-limits-of-integration}}

For constants \(a < b\) and \(c < d\) we can accomodate a change of
finite interval via: \[ \int_{a}^{b}f(x)dx = \int_{c}^{d}g(y)dy, \]
defining
\[g(y) := \frac{b-a}{d-c}g \Big{(}a + \frac{b-a}{d-c}(y-c)\Big{)}. \]

\hypertarget{integrating-the-interpolating-polynomial-approximation}{%
\subsubsection{Integrating the interpolating polynomial
approximation}\label{integrating-the-interpolating-polynomial-approximation}}

If we have a Lagrange polynomial \(p_{k-1}\) over {[}a,b{]} we have:
\[I(p_{k-1}) = \sum_{i=1}^{k}f(x_{i})\int_{a}^{b}l_{i}(x) .\] We know
\[l_{i}(x) = \prod_{j=1,j \neq i}\frac{x - x_{j}}{x_{i} - x_{j}} \] so
the integral approximation depends only on the choice of interpolating
points. The \(l_{i}\) can be complicated but are able to be integrated
by hand for small \(k\).

\hypertarget{newton-cotes-rules}{%
\subsubsection{Newton-Cotes rules}\label{newton-cotes-rules}}

The \emph{rectangular rule} corresponds to a closed scheme with
\(k = 1\): \[\hat{I}_{\rm rectangular}(f) = (b-a) f(a). \] The
\emph{midpoint rule} corresponds to an open scheme with \(k=1\):
\[\hat{I}_{\rm midpoint}(f) = (b-a) f \left ( \frac{a+b}{2} \right). \]
The \emph{trapezoidal rule} corresponds to a closed scheme with \(k=2\):
\[\hat{I}_{\rm trapezoidal}(f) = \frac{b-a}{2} \{ f(a)+f(b) \}. \]
\emph{Simpson's rule} corresponds to a closed scheme with \(k=3\). We
obtain:
\[\hat{I}_{\rm Simpson}(f) = \frac{b-a}{6} \left \{ f(a) + 4 f \left ( \frac{a+b}{2} \right) + f(b) \right \}. \]

It is possible to obtain a crude bound on the error of these schemes. We
find that often the \emph{midpoint} rule can perform better than the
\emph{trapezoid rule} even though it requires less points.

\hypertarget{composite-rules}{%
\subsubsection{Composite rules}\label{composite-rules}}

When a composite polynomial interpolation is used the approximation is
simply the sum of the integrals on each subinterval. Hence a composite
Newton-Cotes rule is obtained by splitting \([a,b]\) into \(m\)
subintervals and summing the approximate intervals for each subinterval.
\[\hat{I}^m_{\rm rule}(f) = \sum_{i=1}^m \hat{I}_{\rm rule}(f_i), \]
where \(f_{i}\) is \(f\) on \([a + (i-1)h,a+ih]\) and
\(h = \frac{b-a}{m}\).

\hypertarget{gaussian-quadrature}{%
\subsubsection{Gaussian quadrature}\label{gaussian-quadrature}}

Guassian quadrature is similar to Chebyshev points in that it exploits
the freedom of interpolation points ot reduce the error of the
approximate integral in the form: \[ I = \int_{a}^{n}f(x)w(x)dx \] where
\(w\) is continuous and positive on \((a,b)\) and
\(\forall n \in \mathbb{N},\int_{a}^{b}x^{n}w(x)dx\) is finite. Defining
the function space:
\[L_{w}^{2}([a,b]) := \Big{ \{ } f:\int_{a}^{b} f(x)^{2}w(x)dx < \infty \Big{ \} } \]
There exists a unique sequencce of orthogonal polynomials
\(p_{0},p_{1},...\) in \(L_{w}^{2}([a,b])\) that are monic. These can be
found using the \textbf{Gram-Schmidt process}. We choose the
interpolation points as the \(k\) roots of \(p_{k}\) for the Gaussian
quadrature rule.

\hypertarget{practical-algorithms}{%
\subsubsection{Practical algorithms}\label{practical-algorithms}}

The methods above can be enhanced by various adaptations and practical
error estimation. They are designed to estimate bounds on the error and
spend more computational effort in subintervals where the integral is
estimated poorly. Defining a robust algorithm involves specifying change
of variables for dealing with semi-infinite and infinite intervals. The
weight function can also be varied to produce a better approximation.

\hypertarget{multiple-integrals}{%
\subsubsection{Multiple integrals}\label{multiple-integrals}}

Consider an integral over
\(D = [a_{1}, b_{1}] \times ... \times [a_{d}, b_{d}]\):
\[I(f) = \int_{D} f(x_{1},...,x_{d})d(x_{1},...,x_{d}).\] According to
Fubini's Theorem, and letting
\(D' = [a_{2}, b_{2}] \times ... \times [a_{d}, b_{d}]\) can often
rewrite \(I(f)\) as an iterated integral:
\[I(f) = \int_{a_{1}}^{b_{1} \int_{D'} f(x_{1},...,x_{d})d(x_{2},...,x_{d})dx_{1} = \int_{a_{1}}^{b_{1}g(x_{1})dx_{1} \]
We develop a recursive algorithm using an approximation of \(g\) to
approximate \(I(f)\).

\[\hat{I}(f) = \sum_{i=1}^k \hat{g}(x_1^{(i)}) \int_{a_1}^{b_1} \ell_i(x_1) {\rm d}x_1, \]

where \$\hat{g}(x\_\{1\}) = \hat{I}(h\_\{x\_\{1\}\}).

\hypertarget{monte-carlo-integration}{%
\subsection{Monte Carlo integration}\label{monte-carlo-integration}}

Quadrature rules quickly become prohibitively expensive in high
dimensional problems, we consider Monte Carlo algorithms for higher
dimensions.

Letting \((X, \mathcal{X})\) be a measurable space. We have target
probability measure \(\pi: \mathcal{X} \rightarrow [0,1]\) and want to
approximate: \[ \pi(f) := \int_{X}f(x)\pi(dx), \]

where \(f \in L_{1}(X, \pi) = \{f:\pi(|f|) < \infty \}\). This is
equivalent to \(\pi(f)\) is the expectation of \(f(X)\) when
\(X \sim \pi\).

\hypertarget{fundamental-results}{%
\subsubsection{Fundamental results:}\label{fundamental-results}}

Let \((X_{n})_{n \meq 1}\) be a sequence of i.i.d. random variables
distributed according to \(\mu\). Define:
\[S_{n}(f) := \sum_{i=1}^{n}f(X_{i}),\] for \(f \in L_{1}(X, \mu).\)
Then: \[\lim_{n \rightarrow \infty} \frac{1}{n}S_{n}(f) = \mu(f), \]

We call this a Monte Carlo approximation of \(\mu(f)\). This is an
unbiased estimator. The variance of this estimate is
\(\frac{\mu(f^{2}) - \mu(f)^{2}}{n}\).

Central limit theroem: If \(f \in L_{2}(X, \mu)\) then:
\[n^{1/2}\{n^{-1}S_{n}(f) - \mu(f) \} \rightarrow X \sim N(0, \mu(\bar{f}^{2})) ,\]
where \(\bar{f}\) = f - \mu(f).

\hypertarget{error-compared-to-quadrature}{%
\subsubsection{Error compared to
quadrature}\label{error-compared-to-quadrature}}

This i sa slow convergence rate compared to quadrature rules in one
dimension. Dimension does not directly affect convergence however
\(\bar{f}\) can be very large for large \(d\).

\hypertarget{sampling}{%
\subsubsection{Sampling}\label{sampling}}

There are several sampling methods available. If we can simulate random
variates \(\pi\) on a computer then this is called perfect sampling.

Another more applicable is rejection sampling. This applies when one can
sample \(\mu\)-distributed random variates from proposal distribution
\(\mu\) and \(\pi/\mu\) satisfies
\(\sup_{x\in X}\pi(x)/\mu(x) \leq M < \infty\). The algorithm is: *
Sample \(X \sim \mu\)

\begin{itemize}
\tightlist
\item
  With probability \(\frac{1}{M}\frac{\pi(X)}{\mu(X)}\) output X,
  otherwise go to step 1
\end{itemize}

Rejection sampling works best when the proposal distribution is close to
the true distribution but it is not required. In many practical
applications \(M\) is prohibitively large.

We can also do importance sampling and self-normalized importance
sampling.

\end{document}
